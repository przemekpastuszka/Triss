%%%%%%%%%%%%%%%%%%%%%%%%%%%%%%%%%%%%%%%%%%%%%%%%%%%%%%%%%%%%%%%%%

\documentclass[11pt]{article}

%%%%%%%%%%%%%%%%%%%%%%%%%%%%%%%%%%%%%%%%%%%%%%%%%%%%%%%%%%%%%%%%%
\usepackage{polski}
\usepackage{listings}
\usepackage[utf8]{inputenc}
\usepackage[T1]{fontenc}
\usepackage{a4wide}
\usepackage{amsthm}
\usepackage{graphicx}
\usepackage{amsmath,amssymb}
\usepackage{caption}
\usepackage{bm}
\usepackage{color}
%\usepackage{epstopdf}
%%%%%%%%%%%%%%%%%%%%%%%%%%%%%%%%%%%%%%%%%%%%%%%%%%%%%%%%%%%%%%%%%

%%%%%%%%%%%%%%%%%%%%%%%%%%%%%%%%%%%%%%%%%%%%%%%%%%%%%%%%%%%%%%%%%
\newtheorem{definition}{Definicja}
\newtheorem{theorem}{Twierdzenie}
\newtheorem{example}{Przykład}
\newtheorem{conclusion}{Wniosek}
\newtheorem{fact}{Fakt}
%%%%%%%%%%%%%%%%%%%%%%%%%%%%%%%%%%%%%%%%%%%%%%%%%%%%%%%%%%%%%%%%%

%%%%%%%%%%%%%%%%%%%%%%%%%%%%%%%%%%%%%%%%%%%%%%%%%%%%%%%%%%%%%%%%%
\title{\LARGE \textbf{ The Revolution In Search Systems}\\
                               Szkic koncepcyjny
}
\author{Przemysław Pastuszka\\ pastuszka.przemyslaw@gmail.com \and
        Michał Rychlik\\ michal.rychlik21@op.pl \and
        Piotr Olchawa\\ piotrekolchawa@gmail.com}
\date{Wrocław, \today}
%%%%%%%%%%%%%%%%%%%%%%%%%%%%%%%%%%%%%%%%%%%%%%%%%%%%%%%%%%%%%

\begin{document}
    \lstset{language=C++}
    \maketitle
    \thispagestyle{empty}
    \tableofcontents
    \newpage

    %%%%%%%%%%%%%%%%%%%%%%%%%%%%%%%%%%%%%%%%%%%%%%%%%%%%%%%%%%%%%%%%%
    \section{Motywacja}
        Rozważmy następujący scenariusz internetowej giełdy reklam:
        \begin{itemize}
            \item użytkownik U odwiedza stronę internetową X. Strona X chciałaby wyświetlić U sprofilowane dla niego reklamy, mając wiedzę o jego preferencjach.
            \item X wysyła w tym celu zapytanie do serwera pośredniczącego wymianą reklam S.
            \item S przekazuje zapytanie do wszystkich znanych mu reklamodawców $R_{i}$.
            \item Każdy z reklamodawców $R_{i}$ znajduje odpowiednie reklamy dla U i odsyła je do S wraz z ceną, jaką jest w stanie zapłacić za ich wyświetlenie
            \item Mając odpowiedzi, S wybiera najlepszą ofertę i zwraca ją X
            \item X wyświetla reklamy użytkownikowi U
            \item $R_{i}$ wysyła należność pieniężną do S. S pobiera opłatę za pośrednictwo, resztę zapłaty wysyłając do X.
        \end{itemize}
        Wymagania stawiane bazie danych na serwerach reklamodawców $R_{i}$
        \begin{itemize}
            \item czas odpowiedzi 200ms (wymaganie stawiane przez S)
            \item liczba obsługiwanych zapytań na 1 sekundę - 10 000
        \end{itemize}
        Biorąc pod uwagę dodatkowe założenie, że zbiór reklam w bazie nie zmienia się często, najważniejszą operacją wykonywaną na bazie danych jest SELECT.\\
        Nie udało się nam znaleźć bazy o otwartym kodzie spełniającej powyższe wymagania, dlatego postanowiliśmy stworzyć TRISS.
    \section{Obsługiwane operacje}
        \subsection{Obsługiwane typy danych}
            \begin{itemize}
                \item typy numeryczne
                \item typ tekstowy
                \item lista homogeniczna
            \end{itemize}
        \subsection{Operacje na tabelach}
            \begin{itemize}
                \item CREATE ... FROM ... - stwórz tabelę o podanym schemacie i załaduj do niej dane z pliku
                \item DROP - usuń tabelę
                \item (opcjonalnie) MERGE tabR1 tabR2 tabDst - połącz dwie tabele o identycznym schemacie
            \end{itemize}
         \subsection{Zapytania SELECT}
            Termy:
             \begin{itemize}
                \item Dla typów numerycznych i tekstowych:
                    \begin{itemize}
                        \item $a = b$
                        \item $a \neq b$
                    \end{itemize}
                \item Dla typów numerycznych
                    \begin{itemize}
                        \item $a < b$
                        \item $a > b$
                    \end{itemize}
                \item Dla typów listowych
                    \begin{itemize}
                        \item $a$ in list
                        \item $a$ not in list
                    \end{itemize}
            \end{itemize}
            Spójniki logiczne:
            \begin{itemize}
                \item AND
                \item (opcjonalnie) OR
            \end{itemize}
    \section{Sposoby dostępu do bazy danych}
        Na początek planujemy:
        \begin{itemize}
            \item
                dostęp przez API = użycie naszej bazy jako zwykłej biblioteki.\\
                Przykład proponowanego API dla operacji SELECT:
        \begin{lstlisting}
            Query q(tableName);
            q.addConstraint(eq("colA",5));
            q.addConstraint(includes("colB", "abcd")};
            Result res = database.select(q);
        \end{lstlisting}
        \end{itemize}
        W późniejszym terminie:
        \begin{itemize}
            \item Operacje definiowane za pomocą pewnego podzbioru SQL
            \item Dostęp do bazy przez sieć w architekturze klient - serwer.
        \end{itemize}
    \section{Przewidywany harmonogram prac na najbliższe 2 tygodnie}
        \subsection{Termin ukończenia - 16 luty}
            \begin{itemize}
                \item Przygotowanie szkicu
                \item Uzgodnienie szczegółów technicznych:
                    \begin{itemize}
                        \item Standardy pisania kodu
                        \item Używane biblioteki
                    \end{itemize}
                \item Rozpoczęcie prac nad pierwszym z prototypów
            \end{itemize}
        \subsection{Termin ukończenia - 23 luty}
            \begin{itemize}
                \item Zakończenie prac nad pierwszym prototypem
                \item Rozpoczęcie prac nad drugim prototypem
            \end{itemize}
    \section{Prototypowanie}
        Szybkość systemu jest kluczowa, dlatego chcielibyślmy wypróbować dwie różne struktury danych do realizacji powyższych zadań. Postanowiliśmy napisać dla nich prototypy i porównać ich wydajność w praktyce.\\Wspomniane pomysły omówimy na najbliższym spotkaniu.
\end{document}
